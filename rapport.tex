\documentclass[11pt]{article}
\usepackage[utf8]{inputenc}
\usepackage[T1]{fontenc}
\usepackage[a4paper, left=1.2in, right=1.2in, top=1.4in, bottom=1.5in]{geometry}

\usepackage{amsmath}
\usepackage{amsfonts}
\usepackage{amssymb}

\author{Aghilas Y. Boussaa \texttt{<aghilas.boussaa@ens.fr>}}
\title{Report for the project of Semantics and Application to Program Verification}
\begin{document}
\maketitle
\begin{abstract}
  We present our project for the course \textit{Semantics and Application to
    Program Verification}. We have implemented an iterator supporting backward
  analysis and all the requiered domains. We added examples to test these.
\end{abstract}
\section{Global architecture}
The iterator starts by a depth first exploration of the forest of nodes to mark
nodes that will require widening to ensure termination.

As suggested by the handle, the iterator then implements a worklist algorithm.

Once a final value has been calculated for each node, each assertion is checked.
If an assertion fails (i.e. the environment where it is false isn't bottom)
while the \texttt{---backward} option has beeen set. The iterator launches a
backward analyses from the source of the assertion arc assuming the assertion
failed. The assertion of figure~\ref{backward} triggers an error with interval
domain if we don't use backward analysis.
\begin{figure}[h]
\begin{verbatim}
void main () {
        int i = rand(1, 2);
        int j = i * 2;
        if (j >= 3) {
                assert(i > 1); //@OK
        }
}
\end{verbatim}
\caption{No error triggered with \texttt{---backward}}\label{backward}
\end{figure}
\section{Domains}
\subsection{Interval}
An interval is represented, either as $\bot$, $\top$, $[a;+\infty[$, $[a;b]$
    (with $a\leq b$) or $]-\infty;b]$, where $a$ and $b$ are integers. This
forces us to consider a lot of sometimes redundant cases when writing our
functions and may not be the best way to implement it. It, at least, forces us
to truly consider edge cases.

All the operation, except for the modulo, are implemented as in the course. To
implement the modulo, we take into account the fact that, in \textsf{C},
\texttt{a\%b} has the sign of \texttt{a} and is smaller, in absolute value, than
\texttt{b}.
\subsection{Sign}
To represent a sign, we take three booleans: can the value be negative, can it
be null and can it be positive. This makes most operations straightforward to
implement (a join is logical or, a meet a logical and, etc.).

Once again, all the operation apart from the modulo are standard. The remark
from the previous subsection gives us that the sign of \texttt{a\%b} with zero
bit set to true. This domain allows us to treat correctly programs such as the
one in figure~\ref{sign} which can't be handled with intervals.
\begin{figure}[h]
\begin{verbatim}
void main () {
        int i = rand(1, 100);
        int j = rand(-100, -1);
        assert(i * j != 0); // @OK
}
\end{verbatim}
\caption{A use case for the sign domain}\label{sign}
\end{figure}
\subsection{Congruence}
We reprensent a value of the congruence domain by either $\bot$ or two integers
representing the set $a\mathbb{Z}+b$.

Most operations are straightforward implementation of the ideas of Antoine
Miné's course notes\footnote{their was a mistake in the description of the meet,
we sent an e-mail to the author and it was corrected in the last version}. They
are exercises in arithmetic we don't detail here. For the modulo, we treat
precisely the case where we divide by a constant.

For widening, the join will suffice, as the factor before (which is a positive
integer) the $\mathbb{Z}$ decreases each time we make a join, and the constant
is the one join's first argument. We don't have a precise implementation of
narrowing. It is harmless as we don't use it in our iterator.
\subsection{Reduced product of the previous ones}
A value of this domain is a tuple of three values, one for each domain.

Once the reduction function has been implemented, all the operations on the
reduced product can be derived from the operations on the factors.

For the reduction, we start by updating the sign with informations from the two
other domains. We, then, update the interval taking into the sign and finally
the interval and the congruence. As the last modification on the interval can
bring new sign information, we iterate this process until we have a
fixpoint\footnote{we will have one, as, once the sign is fixed (decreasing in a
finite domain), we only have the interaction of the interval and congruence; we
conjecture that we only need two iterations as once the sign is updated for the
second time, the rest won't change}.
\subsection{Polyhedral}
Most operations on the polyhderal domain have their direct counterpart in the
\textsf{Apron} library and the only thing to do is providing the good arguments.

The notable exception to this is the \texttt{guard} function, which isn't mapped
directly to an \textsf{Apron} function. To implement it, we take the meet of the
environment given in arguments and several polyhedron representing the
constraints given by the boolean expression. We build this meet little by
little, instead of building one polyhedron and taking the meet at the end to
treat the example of figure~\ref{poly}. If we just build a polyhedron for
\texttt{i+j!=0}, we get $\top$ and the analyzer triggers an error, while
it doesn't with our design.
\begin{figure}[h]
\begin{verbatim}
void main () {
        int i = rand(1, 10);
        int j = -i;
        assert(i + j == 0); //@OK
}
\end{verbatim}
\caption{On the necessity of being cautious with guard}\label{poly}
\end{figure}
\section{Difficulties}
We encountered two main difficulties. One the one hand, it was rather tedious to
treat all the different edge cases for each operations. In particular the modulo
and division were often hard to implement. On the other hand, debugging the
reduced product was complicated by the fact that we had to know if the bug came
from one of the domain, their interaction or something else.
\end{document}
